\documentclass[11]{article}

\begin{document}
Displaying special characters:
$$ \{a, b, c\} $$ % curly brackets must be escaped to be displayed
$$ \ \$ 35.55 $$ % curly brackets must be escaped to be displayed

left and right functions expands the height of braces - these functions must be used together:
$$ 3\left(\frac{2}{5}\right) $$
$$ 3\left[\frac{2}{5}\right] $$
$$ 3\left\{\frac{2}{5}\right\} $$
$$ \left|\frac{x}{x+1}\right| $$

$$ \left| \frac{x}{x+1} \right. $$ % the character however can be omitted by replacing it with a period
$$ \left. \frac{dx}{dy} \right|_{x = 1} $$ 

tables: %

\begin{tabular}{c|ccccc} 
	$x$ & 1 & 2 & 3 & 4 & 5 \\ 
	\hline % inserts a horizontal line
	$f(x)$ & 10 & 22 & 33 & 44 & 55
\end{tabular}

equation arrays: 
\begin{eqnarray} % an array of equations; math mode is automatically enabled
	5x^3 - 9 &=& x + 3 \\
	4x^3 = 12 \\
	x^3 = 3 \\
	x \approx \pm1.732
\end{eqnarray}

\begin{eqnarray} % the and sign helps to align the equations
	5x^3 - 9 &=& x + 3 \\
	4x^3 &=& 12 \\
	x^3 &=& 3 \\
	x &\approx& \pm1.732
\end{eqnarray}

\begin{eqnarray*} % the asterisk sign hides the number of the equations 
	5x^3 - 9 &=& x + 3 \\
	4x^3 &=& 12 \\
	x^3 &=& 3 \\
	x &\approx& \pm1.732
\end{eqnarray*}

\end{document}